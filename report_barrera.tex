% Options for packages loaded elsewhere
\PassOptionsToPackage{unicode}{hyperref}
\PassOptionsToPackage{hyphens}{url}
%
\documentclass[
]{article}
\usepackage{amsmath,amssymb}
\usepackage{lmodern}
\usepackage{iftex}
\ifPDFTeX
  \usepackage[T1]{fontenc}
  \usepackage[utf8]{inputenc}
  \usepackage{textcomp} % provide euro and other symbols
\else % if luatex or xetex
  \usepackage{unicode-math}
  \defaultfontfeatures{Scale=MatchLowercase}
  \defaultfontfeatures[\rmfamily]{Ligatures=TeX,Scale=1}
\fi
% Use upquote if available, for straight quotes in verbatim environments
\IfFileExists{upquote.sty}{\usepackage{upquote}}{}
\IfFileExists{microtype.sty}{% use microtype if available
  \usepackage[]{microtype}
  \UseMicrotypeSet[protrusion]{basicmath} % disable protrusion for tt fonts
}{}
\makeatletter
\@ifundefined{KOMAClassName}{% if non-KOMA class
  \IfFileExists{parskip.sty}{%
    \usepackage{parskip}
  }{% else
    \setlength{\parindent}{0pt}
    \setlength{\parskip}{6pt plus 2pt minus 1pt}}
}{% if KOMA class
  \KOMAoptions{parskip=half}}
\makeatother
\usepackage{xcolor}
\usepackage[margin=1in]{geometry}
\usepackage{color}
\usepackage{fancyvrb}
\newcommand{\VerbBar}{|}
\newcommand{\VERB}{\Verb[commandchars=\\\{\}]}
\DefineVerbatimEnvironment{Highlighting}{Verbatim}{commandchars=\\\{\}}
% Add ',fontsize=\small' for more characters per line
\usepackage{framed}
\definecolor{shadecolor}{RGB}{248,248,248}
\newenvironment{Shaded}{\begin{snugshade}}{\end{snugshade}}
\newcommand{\AlertTok}[1]{\textcolor[rgb]{0.94,0.16,0.16}{#1}}
\newcommand{\AnnotationTok}[1]{\textcolor[rgb]{0.56,0.35,0.01}{\textbf{\textit{#1}}}}
\newcommand{\AttributeTok}[1]{\textcolor[rgb]{0.77,0.63,0.00}{#1}}
\newcommand{\BaseNTok}[1]{\textcolor[rgb]{0.00,0.00,0.81}{#1}}
\newcommand{\BuiltInTok}[1]{#1}
\newcommand{\CharTok}[1]{\textcolor[rgb]{0.31,0.60,0.02}{#1}}
\newcommand{\CommentTok}[1]{\textcolor[rgb]{0.56,0.35,0.01}{\textit{#1}}}
\newcommand{\CommentVarTok}[1]{\textcolor[rgb]{0.56,0.35,0.01}{\textbf{\textit{#1}}}}
\newcommand{\ConstantTok}[1]{\textcolor[rgb]{0.00,0.00,0.00}{#1}}
\newcommand{\ControlFlowTok}[1]{\textcolor[rgb]{0.13,0.29,0.53}{\textbf{#1}}}
\newcommand{\DataTypeTok}[1]{\textcolor[rgb]{0.13,0.29,0.53}{#1}}
\newcommand{\DecValTok}[1]{\textcolor[rgb]{0.00,0.00,0.81}{#1}}
\newcommand{\DocumentationTok}[1]{\textcolor[rgb]{0.56,0.35,0.01}{\textbf{\textit{#1}}}}
\newcommand{\ErrorTok}[1]{\textcolor[rgb]{0.64,0.00,0.00}{\textbf{#1}}}
\newcommand{\ExtensionTok}[1]{#1}
\newcommand{\FloatTok}[1]{\textcolor[rgb]{0.00,0.00,0.81}{#1}}
\newcommand{\FunctionTok}[1]{\textcolor[rgb]{0.00,0.00,0.00}{#1}}
\newcommand{\ImportTok}[1]{#1}
\newcommand{\InformationTok}[1]{\textcolor[rgb]{0.56,0.35,0.01}{\textbf{\textit{#1}}}}
\newcommand{\KeywordTok}[1]{\textcolor[rgb]{0.13,0.29,0.53}{\textbf{#1}}}
\newcommand{\NormalTok}[1]{#1}
\newcommand{\OperatorTok}[1]{\textcolor[rgb]{0.81,0.36,0.00}{\textbf{#1}}}
\newcommand{\OtherTok}[1]{\textcolor[rgb]{0.56,0.35,0.01}{#1}}
\newcommand{\PreprocessorTok}[1]{\textcolor[rgb]{0.56,0.35,0.01}{\textit{#1}}}
\newcommand{\RegionMarkerTok}[1]{#1}
\newcommand{\SpecialCharTok}[1]{\textcolor[rgb]{0.00,0.00,0.00}{#1}}
\newcommand{\SpecialStringTok}[1]{\textcolor[rgb]{0.31,0.60,0.02}{#1}}
\newcommand{\StringTok}[1]{\textcolor[rgb]{0.31,0.60,0.02}{#1}}
\newcommand{\VariableTok}[1]{\textcolor[rgb]{0.00,0.00,0.00}{#1}}
\newcommand{\VerbatimStringTok}[1]{\textcolor[rgb]{0.31,0.60,0.02}{#1}}
\newcommand{\WarningTok}[1]{\textcolor[rgb]{0.56,0.35,0.01}{\textbf{\textit{#1}}}}
\usepackage{graphicx}
\makeatletter
\def\maxwidth{\ifdim\Gin@nat@width>\linewidth\linewidth\else\Gin@nat@width\fi}
\def\maxheight{\ifdim\Gin@nat@height>\textheight\textheight\else\Gin@nat@height\fi}
\makeatother
% Scale images if necessary, so that they will not overflow the page
% margins by default, and it is still possible to overwrite the defaults
% using explicit options in \includegraphics[width, height, ...]{}
\setkeys{Gin}{width=\maxwidth,height=\maxheight,keepaspectratio}
% Set default figure placement to htbp
\makeatletter
\def\fps@figure{htbp}
\makeatother
\setlength{\emergencystretch}{3em} % prevent overfull lines
\providecommand{\tightlist}{%
  \setlength{\itemsep}{0pt}\setlength{\parskip}{0pt}}
\setcounter{secnumdepth}{5}
\usepackage{booktabs}
\usepackage{longtable}
\usepackage{array}
\usepackage{multirow}
\usepackage{wrapfig}
\usepackage{float}
\usepackage{colortbl}
\usepackage{pdflscape}
\usepackage{tabu}
\usepackage{threeparttable}
\usepackage{threeparttablex}
\usepackage[normalem]{ulem}
\usepackage{makecell}
\usepackage{xcolor}
\usepackage{siunitx}

  \newcolumntype{d}{S[
    input-open-uncertainty=,
    input-close-uncertainty=,
    parse-numbers = false,
    table-align-text-pre=false,
    table-align-text-post=false
  ]}
  
\ifLuaTeX
  \usepackage{selnolig}  % disable illegal ligatures
\fi
\IfFileExists{bookmark.sty}{\usepackage{bookmark}}{\usepackage{hyperref}}
\IfFileExists{xurl.sty}{\usepackage{xurl}}{} % add URL line breaks if available
\urlstyle{same} % disable monospaced font for URLs
\hypersetup{
  pdftitle={Barrera-Osorio et al 2011},
  pdfauthor={Dimitrios \& Eva},
  hidelinks,
  pdfcreator={LaTeX via pandoc}}

\title{Barrera-Osorio et al 2011}
\author{Dimitrios \& Eva}
\date{\(2023-03-09\)}

\begin{document}
\maketitle

{
\setcounter{tocdepth}{2}
\tableofcontents
}
\newpage

\hypertarget{motivation}{%
\section{Motivation}\label{motivation}}

\hypertarget{why-is-this-research-question-relevant}{%
\subsection{Why is this research question
relevant?}\label{why-is-this-research-question-relevant}}

Education in Colombia and other middle-income countries face challenges
such as high dropout rates among low-income students, and the reasons
behind them, such as the high cost of education. Conditional cash
transfers (CCTs) are an evidence-based intervention to increase
participation in education. However, the authors highlight that there is
little variability in the structure of programmes. The paper
investigates if changes in the timing of payments affect the outcomes of
interest: Attendance and re-enrollment. Optimising the structure of CCTs
may contribute to improved education outcomes and reduce disparities in
access to education.

\hypertarget{what-are-the-main-hypotheses}{%
\subsection{What are the main
hypotheses?}\label{what-are-the-main-hypotheses}}

\begin{itemize}
\tightlist
\item
  Null hypothesis 1: \[ \beta_{B} Basic - \beta_{S} Savings = 0 \]
\item
  Alternative hypothesis 1: The savings model will improve outcomes
  compared to the basic programme by relaxing possible savings
  constraints.
\end{itemize}

There are additional hypotheses that are not relevant to our
replication:

\begin{itemize}
\tightlist
\item
  Alternative hypothesis 2: The ``Tertiary'' intervention will improve
  rates of graduation and tertiary enrollment compared to the basic
  programme by providing direct incentives for continuation of
  education. Null: The Tertiary intervention does not produce better
  outcomes than the control condition.
\item
  Alternative hypothesis 3: Any of the three treatments leads to better
  outcomes than no intervention. Null: Neither intervention produces
  better outcomes than the control condition.
\end{itemize}

\hypertarget{data-sources}{%
\section{Data sources}\label{data-sources}}

We investigated performing the replication with the data provided as
part of the lecture. However, we soon discovered that the file did not
contain all required variables, nor was there any meta data or other
information about the variables in the dataset. A brief search revealed
that the data and STATA scripts used to obtain the authors' results are
\href{https://www.openicpsr.org/openicpsr/project/113783/version/V1/view?path=/openicpsr/113783/fcr:versions/V1/AEJApp_2010-0132_Data\&type=folder}{freely
available here}.

For this project, we reference the following files:

\begin{itemize}
\tightlist
\item
  Data file: Public\_Data\_AEJApp\_2010-0132.dta
\item
  STATA script for Table 3: Table\_03\_Attendance.do
\item
  Meta data: AEJApp\_2010-0132\_Data\_ReadMe.pdf
\end{itemize}

\hypertarget{where-does-the-data-come-from-country-time-period-source}{%
\subsection{Where does the data come from (country, time period,
source)?}\label{where-does-the-data-come-from-country-time-period-source}}

Data were collected in San Christobal (``Basic'' and ``Savings''
experiments) and in Suba (``Tertiary'' experiment) and combined from six
different sources:

\begin{itemize}
\tightlist
\item
  SISBEN surveys 2003 and 2004: Baseline data on eligible families
\item
  Programme registration data: Basic information on students
\item
  Administrative records: Enrollment records
\item
  Direct observation in 68 out of 251 schools: Attendance data in last
  quarter of 2005 for 7,158 students.
\item
  Survey in 68 schools: Baseline data collection in 2005.
\item
  Survey in 68 schools: Follow-up in 2006.
\end{itemize}

\hypertarget{what-are-the-key-variables-and-how-are-these-measured}{%
\subsection{What are the key variables and how are these
measured?}\label{what-are-the-key-variables-and-how-are-these-measured}}

Key variables for the replication of Table 3 are the outcome variable,
\emph{at\_msamean}. This measures the percentage of days absent using a
verified attendance measure (see metadata doc) and takes values between
0 and 1 (scale). Another important variable is the clustering variable,
\emph{school\_code}, which has 234 levels (one expression of the
variable for each school in the sample). The treatment indicators,
\emph{T1\_treat}, \emph{T2\_treat}, \emph{T3\_treat}, are binary
variables indicating whether the participant was part of that
intervention (value = 1) or not (value = 0).

Additionally, there is a large number of demographic variables at the
individual and household level, measured either as scale variables or
categorical variables. Details can be found
\href{./AEJApp_2010-0132_Data_ReadMe.pdf}{here}.

\hypertarget{method}{%
\section{Method}\label{method}}

\hypertarget{research-design}{%
\subsection{Research design}\label{research-design}}

The research paper describes three interventions designed to improve
attendance and educational outcomes for students in Colombia.

The first intervention (``basic'') is similar to the
PROGRESA/OPORTUNIDADES program, a conditional cash transfer program in
Mexico that operated from 1997 to 2012. It pays participants 30,000
pesos per month (approximately USD 15) if the child attends at least
80\% of the days in that month. Payments are made bi-monthly through a
dedicated debit card, and students will be removed from the program if
they fail to meet attendance targets or are expelled from school.

The second intervention, called the savings treatment, pays two-thirds
of the monthly amount (20,000 pesos or USD 10) to students' families on
a bi-monthly basis, while the remaining one-third is held in a bank
account. The accumulated funds are then made available to students'
families during the period in which students prepare to enroll for the
next school year, with 100,000 pesos (US\$50) available to them in
December if they reach the attendance target every month.

The third intervention, called the tertiary treatment, incentivizes
students to graduate and matriculate to a higher education institution.
The monthly transfer for good attendance is reduced from 30,000 pesos
per month to 20,000 pesos, but upon graduating, the student earns the
right to receive a transfer of 600,000 pesos (USD 300) if they enroll in
a tertiary institution, and after a year if they fail to enroll upon
graduation.

Students were removed from the program if they fail to meet attendance
targets, fail to matriculate to the next grade twice, or are expelled
from school.

In our replication, we focus on the first and second intervention.

The eligibility criteria for the ``basic'' and ``savings'' experiments
were as follows:

\begin{itemize}
\tightlist
\item
  Children had to have finished grade 5 and be enrolled in grades 6 -
  10.
\item
  The children's families had to be classified into the bottom two
  categories on Colombia's poverty index (SISBEN).
\item
  Only households living in San Cristobal prior to 2004 were eligible to
  participate.
\end{itemize}

The paper investigates differences in enrollment and graduation /
progression to tertiary education for the three treatment groups
compared to untreated controls. Randomization to treatment vs control
group was stratified by location, school public vs private, gender and
grade.

\hypertarget{data-preparation}{%
\subsection{Data preparation}\label{data-preparation}}

We imported the data file from STATA format and prepared it for analysis
by first turning categorical variables into factors. For convenience
when producing graphs, we combined the three treatment indicators into a
single factor variable with four expressions (0 = control group, 1 = T1,
2 = T2, 3 = T3).

We then translated the STATA commands to filter the data in line with
the inclusion criteria:

\begin{itemize}
\tightlist
\item
  Dropping ineligible cases from Suba: Drop if suba == 1
\item
  Keeping only those who were selected for the survey in schools:
  survey\_selected == 1
\item
  Drop if grade is \textless{} 6 or grade is 11
\end{itemize}

\begin{Shaded}
\begin{Highlighting}[]
\CommentTok{\# Generate one variable to capture treatment assignment (T1, T2, control)}

\NormalTok{barrera}\SpecialCharTok{$}\NormalTok{T1T2T3 }\OtherTok{\textless{}{-}} \FunctionTok{case\_when}\NormalTok{(}
\NormalTok{  barrera}\SpecialCharTok{$}\NormalTok{T1\_treat }\SpecialCharTok{==} \DecValTok{1} \SpecialCharTok{\textasciitilde{}} \DecValTok{1}\NormalTok{,}
\NormalTok{  barrera}\SpecialCharTok{$}\NormalTok{T2\_treat }\SpecialCharTok{==} \DecValTok{1} \SpecialCharTok{\textasciitilde{}} \DecValTok{2}\NormalTok{,}
\NormalTok{  barrera}\SpecialCharTok{$}\NormalTok{T3\_treat }\SpecialCharTok{==} \DecValTok{1} \SpecialCharTok{\textasciitilde{}} \DecValTok{3}\NormalTok{,}
\NormalTok{  barrera}\SpecialCharTok{$}\NormalTok{T1\_treat }\SpecialCharTok{==} \DecValTok{0} \SpecialCharTok{\&}\NormalTok{ barrera}\SpecialCharTok{$}\NormalTok{T2\_treat }\SpecialCharTok{==} \DecValTok{0} \SpecialCharTok{\&}\NormalTok{ barrera}\SpecialCharTok{$}\NormalTok{T3\_treat }\SpecialCharTok{==} \DecValTok{0} \SpecialCharTok{\textasciitilde{}} \DecValTok{0}
\NormalTok{)}

\NormalTok{barrera}\SpecialCharTok{$}\NormalTok{T1T2T3 }\OtherTok{\textless{}{-}} \FunctionTok{factor}\NormalTok{(barrera}\SpecialCharTok{$}\NormalTok{T1T2T3, }\AttributeTok{level =} \FunctionTok{c}\NormalTok{(}\DecValTok{0}\NormalTok{, }\DecValTok{1}\NormalTok{, }\DecValTok{2}\NormalTok{, }\DecValTok{3}\NormalTok{), }\AttributeTok{labels =} \FunctionTok{c}\NormalTok{(}\StringTok{"Control"}\NormalTok{, }\StringTok{"Basic (T1)"}\NormalTok{, }\StringTok{"Savings (T2)"}\NormalTok{, }\StringTok{"Tertiary (T3"}\NormalTok{))}


\CommentTok{\# Filtering data in line with the following STATA code operations to reproduce table 3, columns 1{-}3:}
\CommentTok{\# Dropping ineligible cases from Suba: Drop if suba == 1 and grade is \textless{} 9}
\CommentTok{\# drop if suba == 1 \& grade \textless{} 9; }
\CommentTok{\# The above seems to be a mistake in the STATA code: this should be grade \textless{} 6 instead of \textless{} 9. }
\CommentTok{\# Keeping only those who were selected for the survey:}
\CommentTok{\# keep if survey\_selected;}
\CommentTok{\# Drop if they are in grade 11}
\CommentTok{\# Filtered data}

\NormalTok{filtered\_barrera }\OtherTok{\textless{}{-}}\NormalTok{ barrera }\SpecialCharTok{\%\textgreater{}\%} \FunctionTok{filter}\NormalTok{(suba }\SpecialCharTok{==} \DecValTok{0}\NormalTok{, grade }\SpecialCharTok{\textgreater{}=} \DecValTok{6}\NormalTok{, survey\_selected }\SpecialCharTok{==} \DecValTok{1}\NormalTok{, grade }\SpecialCharTok{!=} \DecValTok{11}\NormalTok{)}
\end{Highlighting}
\end{Shaded}

The dataset for our analysis is called \emph{filtered\_barrera}.

\hypertarget{analysis}{%
\subsection{Analysis}\label{analysis}}

\hypertarget{what-are-the-assumptions-of-the-method}{%
\subsubsection{What are the assumptions of the
method?}\label{what-are-the-assumptions-of-the-method}}

The authors initially use simple linear regression to compare treatment
groups. They model the relationship between a dependent variable
(outcome; attendance) and two independent variables (whether participant
is allocated to treatment ``basic'', and whether participant is
allocated to treatment ``savings''.)

The assumptions about the data underlying linear regression are:

\begin{enumerate}
\def\labelenumi{\arabic{enumi}.}
\item
  Linearity: There should be a linear relationship between the
  independent and dependent variables.
\item
  Independence: The observations used in the regression analysis should
  be independent of each other. In other words, the value of one
  observation should not be influenced by the value of another
  observation.
\item
  Homoscedasticity: The variance of the dependent variable should be
  constant across all values of the independent variable(s).
\item
  Normality: The dependent variable should be normally distributed at
  each level of the independent variable(s).
\item
  No multicollinearity: If there are multiple independent variables in
  the regression model, there should be no high correlation between
  these independent variables.
\end{enumerate}

If these assumptions are not met, this can lead to unreliable estimators
(regression coefficients) and / or biased standard errors, i.e.~standard
errors that are systematically smaller or larger than the ``true''
standard error. This means that the relationship between dependent and
independent variables is not estimated correctly by the model.

\hypertarget{are-these-assumptions-plausible-in-this-example}{%
\subsubsection{Are these assumptions plausible in this
example?}\label{are-these-assumptions-plausible-in-this-example}}

We test the assumptions of the simplest regression model using the
procedure detailed
\href{https://godatadrive.com/blog/basic-guide-to-test-assumptions-of-linear-regression-in-r}{here}.

\begin{Shaded}
\begin{Highlighting}[]
\CommentTok{\# Setting up model}
\NormalTok{mod0 }\OtherTok{\textless{}{-}} \FunctionTok{lm}\NormalTok{(}\AttributeTok{data =}\NormalTok{ filtered\_barrera, at\_msamean }\SpecialCharTok{\textasciitilde{}}\NormalTok{ T1\_treat }\SpecialCharTok{+}\NormalTok{ T2\_treat)}
\end{Highlighting}
\end{Shaded}

\begin{Shaded}
\begin{Highlighting}[]
\CommentTok{\# 1. Linearity and 3. heteroskedasticity}
\FunctionTok{plot}\NormalTok{(mod0, }\DecValTok{1}\NormalTok{)}
\end{Highlighting}
\end{Shaded}

\includegraphics{report_barrera_files/figure-latex/unnamed-chunk-1-1.pdf}

The plot is not what we would typically expect if these assumptions were
fulfilled.

\begin{Shaded}
\begin{Highlighting}[]
\CommentTok{\# 2. Independence}
\FunctionTok{durbinWatsonTest}\NormalTok{(mod0)}
\end{Highlighting}
\end{Shaded}

\begin{verbatim}
##  lag Autocorrelation D-W Statistic p-value
##    1     0.004010807      1.990513   0.686
##  Alternative hypothesis: rho != 0
\end{verbatim}

A result for the p-value \textgreater{} 0.05 would suggest we can reject
the Null hypothesis and the assumption is met.

\begin{Shaded}
\begin{Highlighting}[]
\CommentTok{\# 4. Normality}
\FunctionTok{plot}\NormalTok{(mod0, }\DecValTok{3}\NormalTok{)}
\end{Highlighting}
\end{Shaded}

\includegraphics{report_barrera_files/figure-latex/unnamed-chunk-3-1.pdf}

This is again not a typical plot.

Plotting fitted vs actual values for T1.

\begin{Shaded}
\begin{Highlighting}[]
\CommentTok{\# Souce: http://www.sthda.com/english/articles/39{-}regression{-}model{-}diagnostics/161{-}linear{-}regression{-}assumptions{-}and{-}diagnostics{-}in{-}r{-}essentials/}

\NormalTok{model.diag.mod0 }\OtherTok{\textless{}{-}} \FunctionTok{augment}\NormalTok{(mod0)}

\FunctionTok{ggplot}\NormalTok{(model.diag.mod0, }\FunctionTok{aes}\NormalTok{(at\_msamean, T1\_treat)) }\SpecialCharTok{+}
  \FunctionTok{geom\_point}\NormalTok{() }\SpecialCharTok{+}
  \FunctionTok{stat\_smooth}\NormalTok{(}\AttributeTok{method =}\NormalTok{ lm, }\AttributeTok{se =} \ConstantTok{FALSE}\NormalTok{) }\SpecialCharTok{+}
  \FunctionTok{geom\_segment}\NormalTok{(}\FunctionTok{aes}\NormalTok{(}\AttributeTok{xend =}\NormalTok{ at\_msamean, }\AttributeTok{yend =}\NormalTok{ .fitted), }\AttributeTok{color =} \StringTok{"red"}\NormalTok{, }\AttributeTok{linewidth =} \FloatTok{0.3}\NormalTok{)}
\end{Highlighting}
\end{Shaded}

\begin{verbatim}
## `geom_smooth()` using formula = 'y ~ x'
\end{verbatim}

\includegraphics{report_barrera_files/figure-latex/unnamed-chunk-4-1.pdf}

Another atypical plot, but the variance seems the same.

The plots reflect the model set-up, where averages are estimated by
treatment group. There appears to be little variation within groups.
This is further explored graphically below.

\begin{Shaded}
\begin{Highlighting}[]
\CommentTok{\# Plotting the outcome variable}

\CommentTok{\# A boxplot for each group}
\FunctionTok{ggplot}\NormalTok{(filtered\_barrera, }\FunctionTok{aes}\NormalTok{(}\AttributeTok{x =}\NormalTok{ T1T2T3, }\AttributeTok{y =}\NormalTok{ at\_msamean)) }\SpecialCharTok{+}
  \FunctionTok{geom\_boxplot}\NormalTok{() }\SpecialCharTok{+}    \CommentTok{\# Box plot for visualization}
  \FunctionTok{labs}\NormalTok{(}\AttributeTok{x =} \StringTok{"Treatment"}\NormalTok{, }\AttributeTok{y =} \StringTok{"Attendance \%"}\NormalTok{)  }\CommentTok{\# Label the axes}
\end{Highlighting}
\end{Shaded}

\includegraphics{report_barrera_files/figure-latex/outcome_var-1.pdf}

\begin{Shaded}
\begin{Highlighting}[]
\CommentTok{\# Histogram of the outcome variable}

\FunctionTok{hist}\NormalTok{(filtered\_barrera}\SpecialCharTok{$}\NormalTok{at\_msamean)}
\end{Highlighting}
\end{Shaded}

\includegraphics{report_barrera_files/figure-latex/outcome_var-2.pdf}

\begin{Shaded}
\begin{Highlighting}[]
\CommentTok{\# Create separate histograms of at\_msamean for each level of T1T2T3}

\FunctionTok{ggplot}\NormalTok{(filtered\_barrera, }\FunctionTok{aes}\NormalTok{(}\AttributeTok{x =}\NormalTok{ at\_msamean)) }\SpecialCharTok{+}
  \FunctionTok{geom\_histogram}\NormalTok{(}\AttributeTok{binwidth =} \FloatTok{0.07}\NormalTok{, }\AttributeTok{alpha =} \FloatTok{0.5}\NormalTok{, }\AttributeTok{position =} \StringTok{"identity"}\NormalTok{) }\SpecialCharTok{+}
  \FunctionTok{labs}\NormalTok{(}\AttributeTok{x =} \StringTok{"Attendance \%"}\NormalTok{, }\AttributeTok{y =} \StringTok{"Frequency"}\NormalTok{) }\SpecialCharTok{+}
  \FunctionTok{facet\_wrap}\NormalTok{(}\SpecialCharTok{\textasciitilde{}}\NormalTok{T1T2T3, }\AttributeTok{ncol =} \DecValTok{3}\NormalTok{) }\SpecialCharTok{+}
  \FunctionTok{geom\_vline}\NormalTok{(}\AttributeTok{xintercept =} \FloatTok{0.8}\NormalTok{, }\AttributeTok{color =} \StringTok{"red"}\NormalTok{, }\AttributeTok{linetype =} \StringTok{"dashed"}\NormalTok{)}\SpecialCharTok{+}
  \FunctionTok{theme}\NormalTok{(}\AttributeTok{panel.border =} \FunctionTok{element\_rect}\NormalTok{(}\AttributeTok{colour =} \StringTok{"black"}\NormalTok{, }\AttributeTok{fill =} \ConstantTok{NA}\NormalTok{, }\AttributeTok{linewidth =} \DecValTok{1}\NormalTok{),}
        \AttributeTok{panel.spacing =} \FunctionTok{unit}\NormalTok{(}\FloatTok{0.5}\NormalTok{, }\StringTok{"lines"}\NormalTok{))}
\end{Highlighting}
\end{Shaded}

\includegraphics{report_barrera_files/figure-latex/outcome_var-3.pdf}

We can see that a large proportion of each sample has an attendance
record above the 80\% requirement, explaining the skew of the
distribution observed in the box plots. It is therefore unlikely that
the assumptions underlying the linear model are met (although the
normality assumption can be relaxed with sufficiently large samples,
which is the case here).

\hypertarget{model-specifications}{%
\subsection{Model specifications}\label{model-specifications}}

In addition to the above, one violation that should be expected based on
the data is that of independence. It is likely that there are unobserved
characteristics at the school level (e.g.~school culture and rules) that
affect the outcome. The following equations show formally how the
analyses were conceptualised(where i is the individual and j is the
school). While the authors state that standard errors are clustered also
clustered within the individual, the STATA code suggests that the only
clustering variable was the school (\emph{school code}), and this is
what we have replicated below.

Model 1:
\[ y_{ij} = \beta_0 + \beta_{B} Basic_i + \beta_{S} Savings_i + \epsilon_{ij} \]
Model 1 is a simple linear model with only treatment allocation as the
dependent variable, while model 2 also includes a collection of student
and household characteristics.

Model 2:
\[ y_{ij} = \beta_0 + \beta_{B} Basic_i + \beta_{S} Savings_i + \delta X_{ijk} + \theta_{j} + \epsilon_{ij} \]

Model 3 builds on model two, but includes a fixed effect for the school
level. Fixed effect models are used to control for unobserved factors
that affect the outcome variable. In this case, the school was chosen as
the fixed effect.

\hypertarget{descriptive-statistics}{%
\section{Descriptive statistics}\label{descriptive-statistics}}

We do not replicate Table 1. Instead, we show the same information for
the sample in our replication (n=5,799). Note that several factor
variables were treated as continuous in the original paper, and we have
corrected this here, showing proportions instead.

\begin{Shaded}
\begin{Highlighting}[]
\NormalTok{table1 }\OtherTok{\textless{}{-}} \FunctionTok{table1}\NormalTok{(}\SpecialCharTok{\textasciitilde{}}\NormalTok{ f\_teneviv }\SpecialCharTok{+}\NormalTok{ s\_utilities }\SpecialCharTok{+}\NormalTok{ s\_durables }\SpecialCharTok{+}\NormalTok{ s\_infraest\_hh }\SpecialCharTok{+}\NormalTok{ s\_age\_sorteo }\SpecialCharTok{+}\NormalTok{ f\_sexo }\SpecialCharTok{+}\NormalTok{ s\_yrs }\SpecialCharTok{+}\NormalTok{ f\_single }\SpecialCharTok{+}\NormalTok{ s\_edadhead }\SpecialCharTok{+}\NormalTok{ s\_yrshead }\SpecialCharTok{+}\NormalTok{ s\_tpersona }\SpecialCharTok{+}\NormalTok{ s\_num18 }\SpecialCharTok{+}\NormalTok{ f\_estrato }\SpecialCharTok{+}\NormalTok{ s\_puntaje }\SpecialCharTok{+}\NormalTok{ s\_ingtotal }\SpecialCharTok{|} \ErrorTok{\textasciitilde{}} \FunctionTok{factor}\NormalTok{(T1T2T3), }\AttributeTok{data=}\NormalTok{filtered\_barrera)}

\NormalTok{table1}
\end{Highlighting}
\end{Shaded}

\begin{tabular}[t]{lllll}
\toprule
  & Control & Basic (T1) & Savings (T2) & Overall\\
\midrule
 & (N=2096) & (N=1895) & (N=1808) & (N=5799)\\
\addlinespace[0.3em]
\multicolumn{5}{l}{\textbf{House posession}}\\
\hspace{1em}Rented & 1160 (55.3\%) & 1013 (53.5\%) & 992 (54.9\%) & 3165 (54.6\%)\\
\hspace{1em}Mortgaged & 150 (7.2\%) & 133 (7.0\%) & 141 (7.8\%) & 424 (7.3\%)\\
\hspace{1em}Owned outright & 570 (27.2\%) & 524 (27.7\%) & 503 (27.8\%) & 1597 (27.5\%)\\
\hspace{1em}Other & 216 (10.3\%) & 225 (11.9\%) & 172 (9.5\%) & 613 (10.6\%)\\
\addlinespace[0.3em]
\multicolumn{5}{l}{\textbf{Utilities}}\\
\hspace{1em}Mean (SD) & 4.64 (1.42) & 4.62 (1.40) & 4.69 (1.39) & 4.65 (1.40)\\
\hspace{1em}Median [Min, Max] & 5.00 [1.00, 6.00] & 5.00 [1.00, 6.00] & 5.00 [1.00, 6.00] & 5.00 [1.00, 6.00]\\
\addlinespace[0.3em]
\multicolumn{5}{l}{\textbf{Index of durable goods}}\\
\hspace{1em}Mean (SD) & 1.35 (0.883) & 1.32 (0.881) & 1.39 (0.871) & 1.35 (0.879)\\
\hspace{1em}Median [Min, Max] & 1.00 [0, 4.00] & 1.00 [0, 4.00] & 1.00 [0, 4.00] & 1.00 [0, 4.00]\\
\addlinespace[0.3em]
\multicolumn{5}{l}{\textbf{Physical infrastructure index of house}}\\
\hspace{1em}Mean (SD) & 11.6 (1.75) & 11.5 (1.82) & 11.7 (1.63) & 11.6 (1.74)\\
\hspace{1em}Median [Min, Max] & 12.0 [3.00, 19.0] & 12.0 [3.00, 18.0] & 12.0 [3.00, 17.0] & 12.0 [3.00, 19.0]\\
\addlinespace[0.3em]
\multicolumn{5}{l}{\textbf{Age}}\\
\hspace{1em}Mean (SD) & 14.1 (5.42) & 14.2 (5.56) & 13.9 (5.04) & 14.1 (5.35)\\
\hspace{1em}Median [Min, Max] & 13.0 [4.00, 72.0] & 13.0 [1.00, 76.0] & 13.0 [3.00, 78.0] & 13.0 [1.00, 78.0]\\
\addlinespace[0.3em]
\multicolumn{5}{l}{\textbf{Gender}}\\
\hspace{1em}Female & 1055 (50.3\%) & 931 (49.1\%) & 916 (50.7\%) & 2902 (50.0\%)\\
\hspace{1em}Male & 1041 (49.7\%) & 964 (50.9\%) & 892 (49.3\%) & 2897 (50.0\%)\\
\addlinespace[0.3em]
\multicolumn{5}{l}{\textbf{Years of Education}}\\
\hspace{1em}Mean (SD) & 5.34 (1.72) & 5.27 (1.70) & 5.29 (1.69) & 5.30 (1.70)\\
\hspace{1em}Median [Min, Max] & 5.00 [0, 14.0] & 5.00 [0, 16.0] & 5.00 [0, 12.0] & 5.00 [0, 16.0]\\
\addlinespace[0.3em]
\multicolumn{5}{l}{\textbf{Single parent household}}\\
\hspace{1em}No & 1492 (71.2\%) & 1334 (70.4\%) & 1270 (70.2\%) & 4096 (70.6\%)\\
\hspace{1em}Yes & 604 (28.8\%) & 561 (29.6\%) & 538 (29.8\%) & 1703 (29.4\%)\\
\addlinespace[0.3em]
\multicolumn{5}{l}{\textbf{Age of Jefe del Hogar}}\\
\hspace{1em}Mean (SD) & 45.6 (10.3) & 45.5 (9.74) & 45.8 (9.80) & 45.6 (9.97)\\
\hspace{1em}Median [Min, Max] & 43.0 [19.0, 91.0] & 43.0 [23.0, 98.0] & 44.0 [24.0, 84.0] & 43.0 [19.0, 98.0]\\
\addlinespace[0.3em]
\multicolumn{5}{l}{\textbf{Years of Education Jefe}}\\
\hspace{1em}Mean (SD) & 5.59 (2.92) & 5.56 (2.80) & 5.49 (2.89) & 5.55 (2.87)\\
\hspace{1em}Median [Min, Max] & 5.00 [0, 22.0] & 5.00 [0, 15.0] & 5.00 [0, 16.0] & 5.00 [0, 22.0]\\
\addlinespace[0.3em]
\multicolumn{5}{l}{\textbf{Number of people in the household}}\\
\hspace{1em}Mean (SD) & 5.40 (1.94) & 5.44 (1.93) & 5.41 (1.93) & 5.42 (1.93)\\
\hspace{1em}Median [Min, Max] & 5.00 [2.00, 19.0] & 5.00 [2.00, 19.0] & 5.00 [2.00, 19.0] & 5.00 [2.00, 19.0]\\
\addlinespace[0.3em]
\multicolumn{5}{l}{\textbf{Number of kids 18 and under}}\\
\hspace{1em}Mean (SD) & 2.63 (1.32) & 2.71 (1.35) & 2.66 (1.33) & 2.67 (1.33)\\
\hspace{1em}Median [Min, Max] & 2.00 [0, 11.0] & 3.00 [0, 12.0] & 2.00 [0, 12.0] & 2.00 [0, 12.0]\\
\addlinespace[0.3em]
\multicolumn{5}{l}{\textbf{Estrato classification}}\\
\hspace{1em}Class 0 & 440 (21.0\%) & 408 (21.5\%) & 379 (21.0\%) & 1227 (21.2\%)\\
\hspace{1em}Class 1 & 292 (13.9\%) & 256 (13.5\%) & 267 (14.8\%) & 815 (14.1\%)\\
\hspace{1em}Class 2 & 1364 (65.1\%) & 1231 (65.0\%) & 1162 (64.3\%) & 3757 (64.8\%)\\
\addlinespace[0.3em]
\multicolumn{5}{l}{\textbf{SISBEN score}}\\
\hspace{1em}Mean (SD) & 11.7 (4.64) & 11.5 (4.51) & 11.5 (4.52) & 11.6 (4.56)\\
\hspace{1em}Median [Min, Max] & 12.4 [1.92, 21.9] & 12.4 [2.28, 21.8] & 12.3 [1.82, 22.0] & 12.3 [1.82, 22.0]\\
\addlinespace[0.3em]
\multicolumn{5}{l}{\textbf{Household Income}}\\
\hspace{1em}Mean (SD) & 367 (239) & 358 (240) & 368 (226) & 364 (235)\\
\hspace{1em}Median [Min, Max] & 332 [0, 3320] & 330 [0, 4000] & 332 [0, 1730] & 332 [0, 4000]\\
\bottomrule
\end{tabular}

\hypertarget{results}{%
\section{Results}\label{results}}

We replicate Table 3, columns 1-3 using the ``feols()'' function which
allows us to specify linear models with fixed effects and clustered
standard errors. We test the Null hypothesis
\[ \beta_{B} Basic - \beta_{S} Savings = 0 \].

\begin{Shaded}
\begin{Highlighting}[]
\CommentTok{\# Model 1}

\CommentTok{\# Source https://evalf21.classes.andrewheiss.com/example/standard{-}errors/}

\NormalTok{feols\_m1 }\OtherTok{\textless{}{-}} \FunctionTok{feols}\NormalTok{(}\AttributeTok{data =}\NormalTok{ filtered\_barrera,}
\NormalTok{               at\_msamean }\SpecialCharTok{\textasciitilde{}}\NormalTok{ T1\_treat }\SpecialCharTok{+}\NormalTok{ T2\_treat,}
               \AttributeTok{cluster =} \SpecialCharTok{\textasciitilde{}}\NormalTok{ school\_code)}

\CommentTok{\# Model 2}

\NormalTok{feols\_m2 }\OtherTok{\textless{}{-}} \FunctionTok{feols}\NormalTok{(}\AttributeTok{data =}\NormalTok{ filtered\_barrera, }
\NormalTok{                  at\_msamean }\SpecialCharTok{\textasciitilde{}}\NormalTok{ T1\_treat }\SpecialCharTok{+}\NormalTok{ T2\_treat }\SpecialCharTok{+}\NormalTok{ f\_teneviv }\SpecialCharTok{+}\NormalTok{ s\_utilities }\SpecialCharTok{+}\NormalTok{ s\_durables }\SpecialCharTok{+}\NormalTok{ s\_infraest\_hh }\SpecialCharTok{+}\NormalTok{ s\_age\_sorteo }\SpecialCharTok{+}\NormalTok{ s\_age\_sorteo2 }\SpecialCharTok{+}\NormalTok{ s\_years\_back }\SpecialCharTok{+}\NormalTok{ s\_sexo }\SpecialCharTok{+}\NormalTok{ f\_estcivil }\SpecialCharTok{+}\NormalTok{ s\_single }\SpecialCharTok{+}\NormalTok{ s\_edadhead }\SpecialCharTok{+}\NormalTok{ s\_yrshead }\SpecialCharTok{+}\NormalTok{ s\_tpersona }\SpecialCharTok{+}\NormalTok{ s\_num18 }\SpecialCharTok{+}\NormalTok{ f\_estrato }\SpecialCharTok{+}\NormalTok{ s\_puntaje }\SpecialCharTok{+}\NormalTok{ s\_ingtotal }\SpecialCharTok{+}\NormalTok{ f\_grade }\SpecialCharTok{+}\NormalTok{ suba }\SpecialCharTok{+}\NormalTok{ s\_over\_age,}
                  \AttributeTok{cluster =} \SpecialCharTok{\textasciitilde{}}\NormalTok{ school\_code)}
\end{Highlighting}
\end{Shaded}

\begin{verbatim}
## The variables 'f_grade10', 'f_grade11' and 'suba' have been removed because of collinearity (see $collin.var).
\end{verbatim}

\begin{Shaded}
\begin{Highlighting}[]
\CommentTok{\# Model 3}
\NormalTok{feols\_m3 }\OtherTok{\textless{}{-}} \FunctionTok{feols}\NormalTok{(}\AttributeTok{data =}\NormalTok{ filtered\_barrera, }
\NormalTok{                  at\_msamean }\SpecialCharTok{\textasciitilde{}}\NormalTok{ T1\_treat }\SpecialCharTok{+}\NormalTok{ T2\_treat }\SpecialCharTok{+}\NormalTok{ f\_teneviv }\SpecialCharTok{+}\NormalTok{ s\_utilities }\SpecialCharTok{+}\NormalTok{ s\_durables }\SpecialCharTok{+}\NormalTok{ s\_infraest\_hh }\SpecialCharTok{+}\NormalTok{ s\_age\_sorteo }\SpecialCharTok{+}\NormalTok{ s\_age\_sorteo2 }\SpecialCharTok{+}\NormalTok{ s\_years\_back }\SpecialCharTok{+}\NormalTok{ s\_sexo }\SpecialCharTok{+}\NormalTok{ f\_estcivil }\SpecialCharTok{+}\NormalTok{ s\_single }\SpecialCharTok{+}\NormalTok{ s\_edadhead }\SpecialCharTok{+}\NormalTok{ s\_yrshead }\SpecialCharTok{+}\NormalTok{ s\_tpersona }\SpecialCharTok{+}\NormalTok{ s\_num18 }\SpecialCharTok{+}\NormalTok{ f\_estrato }\SpecialCharTok{+}\NormalTok{ s\_puntaje }\SpecialCharTok{+}\NormalTok{ s\_ingtotal }\SpecialCharTok{+}\NormalTok{ f\_grade }\SpecialCharTok{+}\NormalTok{ suba }\SpecialCharTok{+}\NormalTok{ s\_over\_age }\SpecialCharTok{|}\NormalTok{ school\_code,}
                  \AttributeTok{cluster =} \SpecialCharTok{\textasciitilde{}}\NormalTok{ school\_code)}
\end{Highlighting}
\end{Shaded}

\begin{verbatim}
## The variables 'f_grade10', 'f_grade11' and 'suba' have been removed because of collinearity (see $collin.var).
\end{verbatim}

\begin{Shaded}
\begin{Highlighting}[]
\CommentTok{\# Specify hypothesis tests}

\NormalTok{hyp1 }\OtherTok{\textless{}{-}} \FunctionTok{linearHypothesis}\NormalTok{(feols\_m1, }\StringTok{"T1\_treat {-} T2\_treat = 0"}\NormalTok{)}
\NormalTok{hyp2 }\OtherTok{\textless{}{-}} \FunctionTok{linearHypothesis}\NormalTok{(feols\_m2, }\StringTok{"T1\_treat {-} T2\_treat"}\NormalTok{)}
\NormalTok{hyp3 }\OtherTok{\textless{}{-}} \FunctionTok{linearHypothesis}\NormalTok{(feols\_m3, }\StringTok{"T1\_treat {-} T2\_treat"}\NormalTok{)}

\CommentTok{\# Save results and format separately from coefficients.}
\NormalTok{chi1 }\OtherTok{\textless{}{-}} \FunctionTok{format}\NormalTok{(}\FunctionTok{round}\NormalTok{(hyp1[,}\DecValTok{2}\NormalTok{][}\DecValTok{2}\NormalTok{], }\DecValTok{2}\NormalTok{), }\AttributeTok{nsmall =} \DecValTok{2}\NormalTok{)}
\NormalTok{chi2 }\OtherTok{\textless{}{-}} \FunctionTok{format}\NormalTok{(}\FunctionTok{round}\NormalTok{(hyp2[,}\DecValTok{2}\NormalTok{][}\DecValTok{2}\NormalTok{], }\DecValTok{2}\NormalTok{), }\AttributeTok{nsmall =} \DecValTok{2}\NormalTok{)}
\NormalTok{chi3 }\OtherTok{\textless{}{-}} \FunctionTok{format}\NormalTok{(}\FunctionTok{round}\NormalTok{(hyp3[,}\DecValTok{2}\NormalTok{][}\DecValTok{2}\NormalTok{], }\DecValTok{2}\NormalTok{), }\AttributeTok{nsmall =} \DecValTok{2}\NormalTok{)}
\NormalTok{p1 }\OtherTok{\textless{}{-}} \FunctionTok{format}\NormalTok{(}\FunctionTok{round}\NormalTok{(hyp1[,}\DecValTok{3}\NormalTok{][}\DecValTok{2}\NormalTok{], }\DecValTok{2}\NormalTok{), }\AttributeTok{nsmall =} \DecValTok{2}\NormalTok{)}
\NormalTok{p2 }\OtherTok{\textless{}{-}} \FunctionTok{format}\NormalTok{(}\FunctionTok{round}\NormalTok{(hyp2[,}\DecValTok{3}\NormalTok{][}\DecValTok{2}\NormalTok{], }\DecValTok{2}\NormalTok{), }\AttributeTok{nsmall =} \DecValTok{2}\NormalTok{)}
\NormalTok{p3 }\OtherTok{\textless{}{-}} \FunctionTok{format}\NormalTok{(}\FunctionTok{round}\NormalTok{(hyp3[,}\DecValTok{3}\NormalTok{][}\DecValTok{2}\NormalTok{], }\DecValTok{2}\NormalTok{), }\AttributeTok{nsmall =} \DecValTok{2}\NormalTok{)}

\CommentTok{\# Defining additional rows for the table output}

\NormalTok{rows }\OtherTok{\textless{}{-}} \FunctionTok{tribble}\NormalTok{(}\SpecialCharTok{\textasciitilde{}}\NormalTok{term, }\SpecialCharTok{\textasciitilde{}}\StringTok{"(1)"}\NormalTok{, }\SpecialCharTok{\textasciitilde{}}\StringTok{"(2)"}\NormalTok{, }\SpecialCharTok{\textasciitilde{}}\StringTok{"(3)"}\NormalTok{,}
                \StringTok{"Chi{-}squared"}\NormalTok{, chi1, chi2, chi3,}
                \StringTok{"p{-}value"}\NormalTok{, p1, p2, p3)}
\FunctionTok{attr}\NormalTok{(rows, }\StringTok{"position"}\NormalTok{) }\OtherTok{\textless{}{-}} \FunctionTok{c}\NormalTok{(}\DecValTok{5}\NormalTok{,}\DecValTok{6}\NormalTok{)}



\CommentTok{\# Combining all model outputs into one table and showing only coefficients on T1\_treat and T2\_treat.}
\CommentTok{\# Adding grouping label to hypothesis test results and grouped column header, also footnotes.}

\FunctionTok{modelsummary}\NormalTok{(}\FunctionTok{list}\NormalTok{(feols\_m1, feols\_m2, feols\_m3) }\SpecialCharTok{\%\textgreater{}\%}
               \FunctionTok{setNames}\NormalTok{(}\FunctionTok{c}\NormalTok{(}\StringTok{"Model 1"}\NormalTok{, }\StringTok{"Model 2"}\NormalTok{, }\StringTok{"Model 3"}\NormalTok{)),}
             \AttributeTok{coef\_omit =} \SpecialCharTok{{-}}\FunctionTok{c}\NormalTok{(}\DecValTok{2}\NormalTok{,}\DecValTok{3}\NormalTok{),}
             \AttributeTok{gof\_omit =} \StringTok{"AIC|BIC|RMSE|R2 W|R2 A"}\NormalTok{,}
             \AttributeTok{stars =} \ConstantTok{TRUE}\NormalTok{,}
             \AttributeTok{add\_rows =}\NormalTok{ rows,}
             \AttributeTok{coef\_rename =} \FunctionTok{c}\NormalTok{(}\StringTok{"Basic treatment"}\NormalTok{,}\StringTok{"Savings treatment"}\NormalTok{),}
             \AttributeTok{title =} \StringTok{"Table 3 {-} Effects on Monitored School Attendance Rates"}\NormalTok{,}
\NormalTok{             ) }\SpecialCharTok{|\textgreater{}}
             \FunctionTok{kable\_styling}\NormalTok{(}\AttributeTok{latex\_options =} \StringTok{"striped"}\NormalTok{) }\SpecialCharTok{|\textgreater{}} \FunctionTok{pack\_rows}\NormalTok{(}\AttributeTok{index =} \FunctionTok{c}\NormalTok{(}\StringTok{" "} \OtherTok{=} \DecValTok{4}\NormalTok{, }\StringTok{"Hypothesis: Basic {-} Savings"} \OtherTok{=} \DecValTok{2}\NormalTok{)) }\SpecialCharTok{|\textgreater{}}
                                                        \FunctionTok{add\_header\_above}\NormalTok{(}\FunctionTok{c}\NormalTok{(}\StringTok{" "} \OtherTok{=} \DecValTok{1}\NormalTok{, }\StringTok{"Basic {-} Savings"} \OtherTok{=} \DecValTok{3}\NormalTok{)) }\SpecialCharTok{|\textgreater{}}
                                                                           \FunctionTok{footnote}\NormalTok{(}\AttributeTok{general =} \StringTok{"We\textquotesingle{}ve created a footnote."}\NormalTok{,}
                                                                                    \AttributeTok{number =} \FunctionTok{c}\NormalTok{(}\StringTok{"Footnote 1"}\NormalTok{, }\StringTok{"Footnote 2"}\NormalTok{))}
\end{Highlighting}
\end{Shaded}

\begin{table}

\caption{\label{tab:models}Table 3 - Effects on Monitored School Attendance Rates}
\centering
\begin{tabular}[t]{lccc}
\toprule
\multicolumn{1}{c}{ } & \multicolumn{3}{c}{Basic - Savings} \\
\cmidrule(l{3pt}r{3pt}){2-4}
  & Model 1 & Model 2 & Model 3\\
\midrule
\cellcolor{gray!6}{Basic treatment} & \cellcolor{gray!6}{\num{0.033}***} & \cellcolor{gray!6}{\num{0.032}***} & \cellcolor{gray!6}{\num{0.032}***}\\
 & (\num{0.007}) & (\num{0.008}) & (\num{0.007})\\
\cellcolor{gray!6}{Savings treatment} & \cellcolor{gray!6}{\num{0.029}**} & \cellcolor{gray!6}{\num{0.027}**} & \cellcolor{gray!6}{\num{0.027}***}\\
 & (\num{0.008}) & (\num{0.008}) & (\num{0.007})\\
\midrule
\addlinespace[0.3em]
\multicolumn{4}{l}{\textbf{Hypothesis: Basic - Savings}}\\
\hspace{1em}\cellcolor{gray!6}{Chi-squared} & \cellcolor{gray!6}{0.31} & \cellcolor{gray!6}{0.40} & \cellcolor{gray!6}{0.48}\\
\hspace{1em}p-value & 0.58 & 0.52 & 0.49\\
\cellcolor{gray!6}{Num.Obs.} & \cellcolor{gray!6}{\num{5799}} & \cellcolor{gray!6}{\num{5799}} & \cellcolor{gray!6}{\num{5799}}\\
R2 & \num{0.003} & \num{0.037} & \num{0.089}\\
\cellcolor{gray!6}{Std.Errors} & \cellcolor{gray!6}{by: school\_code} & \cellcolor{gray!6}{by: school\_code} & \cellcolor{gray!6}{by: school\_code}\\
FE: school_code &  &  & X\\
\bottomrule
\multicolumn{4}{l}{\rule{0pt}{1em}\textit{Note: }}\\
\multicolumn{4}{l}{\rule{0pt}{1em}We've created a footnote.}\\
\multicolumn{4}{l}{\rule{0pt}{1em}\textsuperscript{1} Footnote 1}\\
\multicolumn{4}{l}{\rule{0pt}{1em}\textsuperscript{2} Footnote 2}\\
\multicolumn{4}{l}{\rule{0pt}{1em}+ p $<$ 0.1, * p $<$ 0.05, ** p $<$ 0.01, *** p $<$ 0.001}\\
\end{tabular}
\end{table}

Our models replicate Table 3 in terms of model coefficients, p-values,
standard errors and R-squared.

\hypertarget{replication-of-figure-1}{%
\section{Replication of Figure 1}\label{replication-of-figure-1}}

We replicated Figure 1 because there was no code available for this from
the repository. However, the description of the method in the paper was
not sufficient to arrive at the same figure, as the authors applied
``local polynomial regressions (bandwith = 0.075)''. We use a linear
model to approximate their results.

\begin{Shaded}
\begin{Highlighting}[]
\CommentTok{\#plot}

\NormalTok{barrera}\SpecialCharTok{$}\NormalTok{T1T2T3 }\OtherTok{\textless{}{-}} \FunctionTok{case\_when}\NormalTok{(}
\NormalTok{  barrera}\SpecialCharTok{$}\NormalTok{T1\_treat }\SpecialCharTok{==} \DecValTok{1} \SpecialCharTok{\textasciitilde{}} \DecValTok{1}\NormalTok{,}
\NormalTok{  barrera}\SpecialCharTok{$}\NormalTok{T2\_treat }\SpecialCharTok{==} \DecValTok{1} \SpecialCharTok{\textasciitilde{}} \DecValTok{2}\NormalTok{,}
\NormalTok{  barrera}\SpecialCharTok{$}\NormalTok{T3\_treat }\SpecialCharTok{==} \DecValTok{1} \SpecialCharTok{\textasciitilde{}} \DecValTok{3}\NormalTok{,}
\NormalTok{  barrera}\SpecialCharTok{$}\NormalTok{T1\_treat }\SpecialCharTok{==} \DecValTok{0} \SpecialCharTok{\&}\NormalTok{ barrera}\SpecialCharTok{$}\NormalTok{T2\_treat }\SpecialCharTok{==} \DecValTok{0} \SpecialCharTok{\&}\NormalTok{ barrera}\SpecialCharTok{$}\NormalTok{T3\_treat }\SpecialCharTok{==} \DecValTok{0} \SpecialCharTok{\textasciitilde{}} \DecValTok{0}
\NormalTok{)}


\NormalTok{plot }\OtherTok{\textless{}{-}} \FunctionTok{ggplot}\NormalTok{(}\AttributeTok{data=}\NormalTok{filtered\_barrera, }\FunctionTok{aes}\NormalTok{(}\AttributeTok{x=}\NormalTok{at\_baseline, }\AttributeTok{y=}\NormalTok{at\_msamean, }\AttributeTok{color=}\FunctionTok{factor}\NormalTok{(T1T2T3))) }\SpecialCharTok{+}
  \FunctionTok{geom\_smooth}\NormalTok{(}\AttributeTok{method=}\StringTok{"lm"}\NormalTok{, }\AttributeTok{se=}\ConstantTok{FALSE}\NormalTok{)}\SpecialCharTok{+}
  \FunctionTok{xlim}\NormalTok{(}\FloatTok{0.65}\NormalTok{, }\FloatTok{0.9}\NormalTok{)}\SpecialCharTok{+}
  \FunctionTok{scale\_color\_discrete}\NormalTok{(}\AttributeTok{name =} \StringTok{"Treatments"}\NormalTok{, }\AttributeTok{labels =} \FunctionTok{c}\NormalTok{(}\StringTok{"Control"}\NormalTok{, }\StringTok{"Basics"}\NormalTok{, }\StringTok{"Savings"}\NormalTok{))}\SpecialCharTok{+}
  \FunctionTok{labs}\NormalTok{(}\AttributeTok{y =} \StringTok{"Actual Attendance "}\NormalTok{, }\AttributeTok{x =} \StringTok{"Predicted Baseline Attendance"}\NormalTok{)}


\NormalTok{plot }\SpecialCharTok{+}  \FunctionTok{ggtitle}\NormalTok{(}\StringTok{"Monitored Attendance by Predicted Attendance Basic{-}Savings Experiment"}\NormalTok{)}
\end{Highlighting}
\end{Shaded}

\begin{verbatim}
## `geom_smooth()` using formula = 'y ~ x'
\end{verbatim}

\begin{verbatim}
## Warning: Removed 694 rows containing non-finite values (`stat_smooth()`).
\end{verbatim}

\includegraphics{report_barrera_files/figure-latex/plot-1.pdf}

\hypertarget{additional-exploration}{%
\section{Additional exploration}\label{additional-exploration}}

Part of the problem with this model is the outcome variable, which has a
ceiling effect - values cannot exceed 100\%, and many people have high
attendance, with target attendance also being high at 80\%.

It may be worthwhile investigating whether there is a significant
difference in the proportion above the cut-off between samples.

\includegraphics{report_barrera_files/figure-latex/expl1-1.pdf} What
proportion in each group falls at or above the target of 80\%
attendance?

\begin{verbatim}
## # A tibble: 3 x 2
##   T1T2T3       prop_cutoff
##   <fct>              <dbl>
## 1 Control            0.765
## 2 Basic (T1)         0.814
## 3 Savings (T2)       0.798
\end{verbatim}

Variability in the outcome variable is limited because most students
attend at least 80\% of the time. An alternative model specification may
be to analyse differences in proportion of attendance (above / below
cut-off).To this end, a binary variable called \emph{above\_cutoff} was
created and a model based on model 3 (but without clustered standard
errors) was run using the glm command.

\begin{Shaded}
\begin{Highlighting}[]
\CommentTok{\# Calculating new column: is participant at or above cut{-}off?}


\NormalTok{filtered\_barrera }\OtherTok{\textless{}{-}}\NormalTok{ filtered\_barrera }\SpecialCharTok{\%\textgreater{}\%} 
  \FunctionTok{mutate}\NormalTok{(}\AttributeTok{above\_cutoff =} \FunctionTok{ifelse}\NormalTok{(at\_msamean }\SpecialCharTok{\textgreater{}=}\NormalTok{ cutoff, }\DecValTok{1}\NormalTok{, }\DecValTok{0}\NormalTok{))}

\CommentTok{\# Running above model but with this as outcome {-}{-}\textgreater{} note that the residual plots don\textquotesingle{}t pick up the clustered standard errors.}

\NormalTok{mod\_bi }\OtherTok{\textless{}{-}} \FunctionTok{glm}\NormalTok{(}\AttributeTok{data =}\NormalTok{ filtered\_barrera, above\_cutoff }\SpecialCharTok{\textasciitilde{}}\NormalTok{ T1\_treat }\SpecialCharTok{+}\NormalTok{ T2\_treat }\SpecialCharTok{+}\NormalTok{ f\_teneviv }\SpecialCharTok{+}\NormalTok{ s\_utilities }\SpecialCharTok{+}\NormalTok{ s\_durables }\SpecialCharTok{+}\NormalTok{ s\_infraest\_hh }\SpecialCharTok{+}\NormalTok{ s\_age\_sorteo }\SpecialCharTok{+}\NormalTok{ s\_age\_sorteo2 }\SpecialCharTok{+}\NormalTok{ s\_years\_back }\SpecialCharTok{+}\NormalTok{ s\_sexo }\SpecialCharTok{+}\NormalTok{ f\_estcivil }\SpecialCharTok{+}\NormalTok{ s\_single }\SpecialCharTok{+}\NormalTok{ s\_edadhead }\SpecialCharTok{+}\NormalTok{ s\_yrshead }\SpecialCharTok{+}\NormalTok{ s\_tpersona }\SpecialCharTok{+}\NormalTok{ s\_num18 }\SpecialCharTok{+}\NormalTok{ f\_estrato }\SpecialCharTok{+}\NormalTok{ s\_puntaje }\SpecialCharTok{+}\NormalTok{ s\_ingtotal }\SpecialCharTok{+}\NormalTok{ f\_grade }\SpecialCharTok{+}\NormalTok{ suba }\SpecialCharTok{+}\NormalTok{ s\_over\_age }\SpecialCharTok{+} \FunctionTok{factor}\NormalTok{(school\_code), }\AttributeTok{family =} \FunctionTok{binomial}\NormalTok{())}

\CommentTok{\# vcov1 \textless{}{-} vcovCL(mod\_bi, cluster = filtered\_barrera$school\_code)}
\CommentTok{\# coeftest(mod\_bi, vcov = vcov1)}


\CommentTok{\# To compare, running linear model as glm:}

\NormalTok{mod\_gau }\OtherTok{\textless{}{-}} \FunctionTok{glm}\NormalTok{(}\AttributeTok{data =}\NormalTok{ filtered\_barrera, at\_msamean }\SpecialCharTok{\textasciitilde{}}\NormalTok{ T1\_treat }\SpecialCharTok{+}\NormalTok{ T2\_treat }\SpecialCharTok{+}\NormalTok{ f\_teneviv }\SpecialCharTok{+}\NormalTok{ s\_utilities }\SpecialCharTok{+}\NormalTok{ s\_durables }\SpecialCharTok{+}\NormalTok{ s\_infraest\_hh }\SpecialCharTok{+}\NormalTok{ s\_age\_sorteo }\SpecialCharTok{+}\NormalTok{ s\_age\_sorteo2 }\SpecialCharTok{+}\NormalTok{ s\_years\_back }\SpecialCharTok{+}\NormalTok{ s\_sexo }\SpecialCharTok{+}\NormalTok{ f\_estcivil }\SpecialCharTok{+}\NormalTok{ s\_single }\SpecialCharTok{+}\NormalTok{ s\_edadhead }\SpecialCharTok{+}\NormalTok{ s\_yrshead }\SpecialCharTok{+}\NormalTok{ s\_tpersona }\SpecialCharTok{+}\NormalTok{ s\_num18 }\SpecialCharTok{+}\NormalTok{ f\_estrato }\SpecialCharTok{+}\NormalTok{ s\_puntaje }\SpecialCharTok{+}\NormalTok{ s\_ingtotal }\SpecialCharTok{+}\NormalTok{ f\_grade }\SpecialCharTok{+}\NormalTok{ suba }\SpecialCharTok{+}\NormalTok{ s\_over\_age }\SpecialCharTok{+} \FunctionTok{factor}\NormalTok{(school\_code), }\AttributeTok{family =} \FunctionTok{gaussian}\NormalTok{())}

\CommentTok{\# vcov2 \textless{}{-} vcovCL(mod\_gau, cluster = filtered\_barrera$school\_code)}
\CommentTok{\# coeftest(mod\_gau, vcov = vcov2)}
\end{Highlighting}
\end{Shaded}

Comparing the coefficients and the Akaike criterion for the two models
and the previously fitted models using ``feols()'' shows that the AIC is
much larger for the logistic regression, indicating a better model fit.
However, the p-values on the treatment coefficients are less convincing,
suggesting that with this better fit, the results would not hold at the
95\% level of confindence (but at the 90\% level, which is more accepted
in economics / policy evaluation than other areas of research).

\begin{verbatim}
## Warning in predict.lm(object, newdata, se.fit, scale = 1, type = if (type == :
## prediction from a rank-deficient fit may be misleading
\end{verbatim}

\begin{table}

\caption{\label{tab:unnamed-chunk-6}Comparison of logistic and linear regressions (no clustered standard errors)}
\centering
\begin{tabular}[t]{lccccc}
\toprule
  & GLM binomial & GLM Gaussian & Model 1 & Model 2 & Model 3\\
\midrule
Basic treatment & \num{0.336}*** & \num{0.032}*** & \num{0.033}*** & \num{0.032}*** & \num{0.032}***\\
 & (\num{0.084}) & (\num{0.008}) & (\num{0.007}) & (\num{0.008}) & (\num{0.007})\\
Savings treatment & \num{0.219}** & \num{0.027}** & \num{0.029}** & \num{0.027}** & \num{0.027}***\\
 & (\num{0.084}) & (\num{0.008}) & (\num{0.008}) & (\num{0.008}) & (\num{0.007})\\
\midrule
AIC & \num{5351.7} & \num{1032.3} & \num{1447.1} & \num{1302.3} & \num{1030.3}\\
\bottomrule
\multicolumn{6}{l}{\rule{0pt}{1em}+ p $<$ 0.1, * p $<$ 0.05, ** p $<$ 0.01, *** p $<$ 0.001}\\
\end{tabular}
\end{table}

How do the two GLM models perform in terms of the distribution of
residuals?

\includegraphics{report_barrera_files/figure-latex/unnamed-chunk-7-1.pdf}

The residuals for the logistic regression appear to be slightly more
``normal'', but this is still not a very good fit.

Another approach may be to try and fit a more parsimonious model to
avoid the risk of colininearity. Many of the covariates appear to be
related to poverty. We therefore test if the explanatory power of the
model is decreased by removing the variables from Model 3 that may be
related to the SISBEN score. We then compare this model to Model 3, and
the logistic regression.

\begin{verbatim}
## The variables 'f_grade10', 'f_grade11' and 'suba' have been removed because of collinearity (see $collin.var).
\end{verbatim}

\begin{verbatim}
## Warning in predict.lm(object, newdata, se.fit, scale = 1, type = if (type == :
## prediction from a rank-deficient fit may be misleading
\end{verbatim}

\begin{table}

\caption{\label{tab:parsimony}Comparison of logistic and linear regressions (no clustered standard errors)}
\centering
\begin{tabular}[t]{lccc}
\toprule
  & GLM binomial & Model 3 & Reduced Model 3\\
\midrule
Basic treatment & \num{0.336}*** & \num{0.032}*** & \num{0.032}***\\
 & (\num{0.084}) & (\num{0.007}) & \vphantom{1} (\num{0.007})\\
Savings treatment & \num{0.219}** & \num{0.027}*** & \num{0.028}***\\
 & (\num{0.084}) & (\num{0.007}) & (\num{0.007})\\
\midrule
AIC & \num{5351.7} & \num{1030.3} & \num{1031.8}\\
\bottomrule
\multicolumn{4}{l}{\rule{0pt}{1em}+ p $<$ 0.1, * p $<$ 0.05, ** p $<$ 0.01, *** p $<$ 0.001}\\
\end{tabular}
\end{table}

\hypertarget{conclusion}{%
\section{Conclusion}\label{conclusion}}

We were able to replicate the model coefficients, p-values, standard
errors and R-squared, despite the fact that R does not have the same
convenient standard options for analysis of clustered data that we find
in STATA.

We observe that there are several errors in the STATA code provided by
the authors.

Further, none of the models (Model 1, Model 2, Model 3) explain more
than 10\% of the variation in the data - depspite the inclusion of a
large number of co-variates and taking into account potential
clustering. It should be noted that there was no attempt by the authors
to optimise model fit / adhering to the principle of parsimony by
removing those variables that are shown not to improve the model.

Our exploratory analysis using logistic regression shows that one issue
may be with the functional form of the model, related to the
distribution of the outcome variable.

Overall, it seems that the analysis could benefit from more thorough
consideration of model choice and model fit, with justifications
provided for the inclusion of each variable. The evidence generated
appears weak due to small values of R-squared and the structural issues
mentioned above.

\end{document}
